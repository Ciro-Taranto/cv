\documentclass[border=10pt]{standalone}
\usepackage[utf8]{inputenc}
\usepackage{tikz}
\usetikzlibrary{mindmap,shadows}
\usetikzlibrary{backgrounds}
\usetikzlibrary{calc}
\usepackage[hidelinks,pdfencoding=auto]{hyperref}


\begin{document}
\definecolor{textcolor}{rgb}{1.0,1.0,1.0}
\definecolor{skillcolor}{rgb}{0.8,0.2,0.2}
\definecolor{hardcolor}{rgb}{0.2,0.2,0.2}
\definecolor{lancolor}{rgb}{0.2,0.2,0.3}
\definecolor{pycolor}{RGB}{153,204,255}
\definecolor{packagecolor}{rgb}{0.2,0.3,0.3}
\def \skillHardDist {7}  
\def \hardLanDist {5} 
\def \lanPacDist {4} 
\tikzset{
		every node/.style=
		{
			concept,circular drop shadow, text=white
    		}
    	}

\begin{tikzpicture}
	[	font = \Large,
		hard/.style={color=hardcolor,text=textcolor, textwidth=4, align=center, minimum size=4cm},
		lan/.style={color=lancolor,text=white,textwidth=3,align=center, minimum size=3cm},
		pac/.style={color=packagecolor,text=white,textwidth=2,align=center, minimum size=2cm},
		gits/.style={color=hardcolor,text=textcolor, textwidth=4, align=center, minimum size=1.5cm},
		layer0/.style={circle connection bar switch color=from (skillcolor) to (hardcolor)},
		layer1/.style={circle connection bar switch color=from (hardcolor) to (lancolor)},
		layer2/.style={circle connection bar switch color=from (lancolor) to (packagecolor)}, 
		layer1py/.style={circle connection bar switch color=from (hardcolor) to (pycolor)},
		layer2py/.style={circle connection bar switch color=from (pycolor) to (packagecolor)},
	]
	%%POSITIONING THE CENTRAL NODE
	\node  at (5,5)  (CS)[minimum size=5cm, color=skillcolor, text=textcolor]{\textbf{Computer skills}};
	%%POSITIONING THE HARD SKILLS NODES%%
	\path (CS) ++(15:\skillHardDist)  node (cal) [hard] {Large scale \\ calculations}; %this command just define the position
	\path (CS) ++(105:\skillHardDist)  node (AV)  [hard] {Data analysis/ \\ visualization};
	\path (CS) ++(235:5.5)  node (git) [gits] {version control \\ git};
	%%POSITIONING THE PROGRAMMING LANGUAGE NODES%%
	\path (AV) ++(0:6) node (py) [lan, inner sep = 1cm,color=pycolor, text=white] {\fontsize{14}{11pt}\textbf{python}}; 
	\path (cal)++(45:\hardLanDist)  node (cpp)[lan] {  {\fontsize{14}{11pt} \textbf{C++}}}; 
	\path (cal)++(315:\hardLanDist)   node (fortran) [lan] {\textbf{Fortran}};
	%%POSITIONING THE PACKAGES NODES %%
	\path (py) ++(135:6.5) node (scikit) [pac] {scikit}; 
	\path (py) ++(118:6.5) node (pandas) [pac] {pandas}; 
	\path (py) ++(75:5)  node (np)     [pac,textwidth=1cm] {numpy \\scipy}; 
	\path (py) ++(30:5)  node (plt)    [pac] {matplotlib};
	\path (fortran) ++(30:\lanPacDist) node (llapack) [pac] {llapack}; 
	\path (fortran) ++(315:\lanPacDist)  node (MPI)     [pac] {MPI}; 
	\path (cpp)    ++(45:\lanPacDist)  node (boost) [pac]{boost}; 
	\path (cpp)    ++(330:\lanPacDist)   node (openMP)[pac]{openMP}; 
	%POSITIONING THE CONNECTING PATHS FOR THE HARD SKILLS%%
	\path (CS)  to [layer0] (git);
	\path (CS)  to [layer0] (cal); 
	\path (CS)  to [layer0] (AV);
	%%POSITIONING THE CONNECTING PATHS FOR THE PROGRAMMING LANGUAGES
	\path (AV) to [layer1py] (py);	
	\path (cal) to [layer1] (fortran); 
	\path (cal) to [layer1] (cpp);
	\path (cal) to [layer1py] (py);
	%%POSITIONING THE CONNECTING PATHS FOR THE PACKAGES
	\path (py) to [layer2py] (scikit); 
	\path (py) to [layer2py] (np); 
	\path (py) to [layer2py] (pandas); 
	\path (py) to [layer2py] (plt); 
	\path (fortran) to [layer2] (llapack);
	\path (fortran) to [layer2] (MPI); 
	\path (cpp) to [layer2] (boost); 
	\path (cpp) to [layer2] (openMP); 
	%%%% guidelines to positon the objects! 
 	%\draw[help lines,step=5mm,gray!20] (0,0) grid (20,20); 
\end{tikzpicture}
\end{document}